\section{Introduction}

\label{intro}

To start on this long way we will focus on the principal key words of this Thesis.

\vspace{5 mm}

\noindent These are: 

\begin{itemize}
\item White Rabbit Project
\begin{itemize}
\item Synchronous Ethernet (SyncE) 
\item PTP (Precision Time Protocol) 
\item Chronos Project
\end{itemize}
\item SPAD Sensors
\item TSN
\item RISC-V.
\item Neuromorphic Sensors
\end{itemize}

\subsection{White Rabbit Project}

The White Rabbit Project \cite{white-rabbit} is a high-precision synchronization system developed by CERN (the European Organization for Nuclear Research).
It’s designed to achieve extremely accurate time synchronization across \textbf{distributed networks}. 
Specifically, it combines Ethernet and Synchronous Ethernet (SyncE) technologies with PTP (Precision Time Protocol), creating what is known as the \say{White Rabbit Protocol}.
The primary goal is to ensure that all connected devices are synchronized within sub-nanosecond accuracy.


\subsubsection{Synchronous Ethernet (SyncE)}

\subsubsection{PTP (Precision Time Protocol)}

\subsubsection{Chronos Project}

Chronos is an advanced timing and synchronization project at CERN, designed to support extremely precise timekeeping requirements in high-energy physics experiments. 

\vspace{5 mm}

\noindent Built on the foundations of the White Rabbit Project, Chronos \cite{gl:chronos} aims to improve upon CERN’s existing timing systems by achieving even more refined synchronization, essential for coordinating complex experiments and data acquisition across CERN’s infrastructure.

\subsection{SPAD Sensors}

SPAD (Single-Photon Avalanche Diode) \cite{9031298} sensors are an advanced type of photodiode sensor, specifically designed to detect extremely low amounts of light, to the single photon level. 
This makes them ideal for applications requiring exceptional sensitivity and the ability to measure extremely \textbf{fast} light events.

\vspace{5 mm}

\noindent Characteristics and Operation of SPADs:

\begin{itemize}
\item High Sensitivity
    \begin{itemize}
    \item[>] SPAD sensors can detect single photons, thanks to their ability to produce an \say{avalanche} of current when a single photon is detected.
    \item[>] This mechanism allows the sensor to significantly amplify the photon signal, providing a clear and strong response.
    \end{itemize}
\item Fast Response Time
    \begin{itemize}
    \item[>] SPADs are capable of recording events in times on the order of picoseconds.
    \item[>] This is useful in applications that require accurate photon arrival time (known as Time-of-Flight or TOF) measurements, such as in LIDAR and Time-Correlated Single Photon Counting (TCSPC), which is used in fluorescence studies.
    \end{itemize}
\item Geiger Mode Operation
    \begin{itemize}
    \item[>] SPADs operate in \say{Geiger} mode, which means that they are polarized at a voltage above the breakdown voltage. 
    \item[>] This causes an electron avalanche when a photon is detected, which allows the detection of the photon to be clearly recorded.
    \item[>] After detection, the sensor \textbf{needs a reset} to be ready for the next event.
    \end{itemize}
\item Dark noise/dark count noise
    \begin{itemize}
    \item[>] Although SPADs are very sensitive, they can suffer from “dark noise”, i.e. false signals caused by thermal electrons in the absence of light. 
    \item[>] This can be significantly controlled with cooling systems or by using signal processing techniques to filter out false events.
    \end{itemize}
\end{itemize}

\vspace{5 mm}

\noindent SPAD Sensor Applications:

\vspace{5 mm}

\noindent SPAD sensors are useful in many areas because of their their photon-level light detection capability and fast response.

\begin{itemize}
\item LIDAR (Light Detection and Ranging)
    \begin{itemize}
    \item[>] SPADs are an ideal choice for high-precision LIDAR systems that measure distances based on the time of flight of reflected photons. 
    \item[>] This is useful in autonomous vehicles, mapping, and environmental monitoring.
    \end{itemize}
\item Biomedicine and Microscopy
    \begin{itemize}
    \item[>] In fluorescence microscopy, SPADs are used for precise photon detection in molecular dynamics studies, where scientists analyze how molecules interact as a function of time.
    \end{itemize}
\item Time-of-Flight (ToF) cameras
    \begin{itemize}
    \item[>] These cameras, which are used in smartphones and augmented reality devices, employ SPAD sensors to capture \textbf{depth information}.
    \item[>] This allows measuring the distance to each point in the image, creating 3-D maps of the environment.
    \end{itemize}
\item Quantum Communications
    \begin{itemize}
    \item[>] SPADs are essential in quantum communications and quantum cryptography, as they can detect single photons that carry quantum information without loss of the signal integrity.
    \end{itemize}
\end{itemize}

\vspace{5 mm}

\noindent Advantages and Disadvantages

\begin{itemize}
\item Advantages:
    \begin{itemize}
    \item[>] Extreme photon sensitivity, appropriate for low light conditions.
    \item[>] Ultra-fast response time, ideal for precise detection and analysis applications.
    \item[>] Quantum information processing capability, important in cryptography and fundamental science.
    \end{itemize}
\item Disadvantages:
    \begin{itemize}
    \item[>] Dark noise, which can limit its accuracy if not properly controlled.
    \item[>] Cost and complexity of manufacture, making them more expensive than other conventional light sensors.
    \item[>] Need for cooling systems in applications where dark noise is critical, which can complicate system design.
    \end{itemize}
\end{itemize}

\subsection{TSN}

\subsection{RISC-V}

\subsection{Neuromorphic Sensors}


