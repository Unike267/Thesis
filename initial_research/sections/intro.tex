\section{Introduction}

\label{intro}

To start on this long way we will focus on the principal key words of this Thesis.

\vspace{5 mm}

\noindent These are: 

\begin{itemize}
\item Spiking Neural Networks
\item SPAD sensors
\item DVS sensors
\item TSN
\item RISC-V.
\end{itemize}

\subsection{Spiking Neural Networks}

\subsection{SPAD sensors}

SPAD (Single-Photon Avalanche Diode) \cite{9031298} sensors are an advanced type of photodiode sensor, specifically designed to detect extremely low amounts of light, to the single photon level. 
This makes them ideal for applications requiring exceptional sensitivity and the ability to measure extremely \textbf{fast} light events.

\vspace{5 mm}

\noindent Characteristics and Operation of SPADs:

\begin{itemize}
\item High Sensitivity
    \begin{itemize}
    \item[*] SPAD sensors can detect single photons, thanks to their ability to produce an \say{avalanche} of current when a single photon is detected.
    \item[*] This mechanism allows the sensor to significantly amplify the photon signal, providing a clear and strong response.
    \end{itemize}
\item Fast Response Time
    \begin{itemize}
    \item[*] SPADs are capable of recording events in times on the order of picoseconds.
    \item[*] This is useful in applications that require accurate photon arrival time (known as Time-of-Flight or TOF) measurements, such as in LIDAR and Time-Correlated Single Photon Counting (TCSPC), which is used in fluorescence studies.
    \end{itemize}
\item Geiger Mode Operation
    \begin{itemize}
    \item[*] SPADs operate in \say{Geiger} mode, which means that they are polarized at a voltage above the breakdown voltage. 
    \item[*] This causes an electron avalanche when a photon is detected, which allows the detection of the photon to be clearly recorded.
    \item[*] After detection, the sensor \textbf{needs a reset} to be ready for the next event.
    \end{itemize}
\item Dark noise/dark count noise
    \begin{itemize}
    \item[*] Although SPADs are very sensitive, they can suffer from “dark noise”, i.e. false signals caused by thermal electrons in the absence of light. 
    \item[*] This can be significantly controlled with cooling systems or by using signal processing techniques to filter out false events.
    \end{itemize}
\end{itemize}

\subsection{DVS sensors}

\subsection{TSN}

\subsection{RISC-V}


